% ----------------------------------------------------------------
% Article Class (This is a LaTeX2e document)  ********************
% ----------------------------------------------------------------
\documentclass[11pt]{article}
\usepackage[english]{babel}
\usepackage{amsmath,amsthm, amssymb}
\usepackage{amsfonts}
\usepackage{multirow}
\usepackage{threeparttable}
\usepackage{enumerate}

%Candelaria's favorite packages
\usepackage{setspace}	% Set the spacing for the document
\usepackage{xtab}		% Use for tables
\usepackage{comment}	% Use for block comments
\usepackage{rotating}	% Helpful for rotating figures
\usepackage{lscape}	% Change a page into landscape
\usepackage{hyperref}
\hypersetup{colorlinks=false,pdfborder={0 0 0},breaklinks=true}
\usepackage{graphicx}	% Load a graphic image
%\usepackage{url}		% Use for formatting URLS (nice for NatBib also)
\usepackage{booktabs}  	% Use for nice tables/ Works well with ESTOUT
\usepackage{longtable} 	% Used to break long tables over multiple pages
\usepackage{caption}
\usepackage{subcaption} 	% Used to create SubFloats
\usepackage{listings} 	% Use to enter code blocks, like a fancy verbatim

% ADD COLOR:
\usepackage{color}
\usepackage[usenames, dvipsnames, svgnames, table]{xcolor}

% Packages for the bibliography
\usepackage[longnamesfirst]{natbib}
%\usepackage{natbib}

% Set Page Margins
%\usepackage{fullpage}
\addtolength{\oddsidemargin}{-.875in}
\addtolength{\evensidemargin}{-.875in}
\addtolength{\textwidth}{1.75in}
\addtolength{\topmargin}{-.875in}
\addtolength{\textheight}{1.75in}

% Set line and table spacing
\renewcommand{\baselinestretch}{1.0}
\renewcommand{\arraystretch}{1.1}
\setlength\abovedisplayskip{0pt}
\setlength\belowdisplayskip{0pt}

% Redefine the cite command
\renewcommand\cite{\citet}

%\setlength{\parindent}{0in}

%\setlength{\leftmargin}{0pt} \setlength{\parskip}{1.1ex} \setlength{\parindent}{1em} \setlength{\itemsep}{0pt} \setlength{\footnotesep}{10pt}
%\renewcommand{\footnoterule}{\rule{3in}{.3pt}\vspace{-.3pt}}

% THEOREMS -------------------------------------------------------
\newtheorem{thm}{Theorem}[section]
\newtheorem{cor}[thm]{Corollary}
\newtheorem{lem}[thm]{Lemma}
\newtheorem{prop}[thm]{Proposition}
\theoremstyle{definition}
\newtheorem{defn}[thm]{Definition}
\theoremstyle{remark}
\newtheorem{rem}[thm]{Remark}
%\numberwithin{equation}{section} %Number equations within section
% ----------------------------------------------------------------



\begin{document}

\begin{center}
{\huge Problem Set 2}\\[5pt]
{\Large Calculus}
\end{center}

\section*{Problem 1: Derivatives}
 Find the derivative of each expression with respect to $x$.
\begin{enumerate}[(a)]
\item $ \displaystyle y=55x$
{\color{red} $$\frac{dy}{dx} = 55 $$}
\item $ \displaystyle y=\frac{1}{3}x^3$
{\color{red} $$ \frac{dy}{dx} = x^2 $$}
\item $ \displaystyle y=(2x)^3$
{\color{red} $$ y = 8x^3 $$
$$\frac{dy}{dx} = 24x^2$$}
\item $ \displaystyle y= 3x^7 - 20x^6 + 16x^5 - 4x^4 + 17x^3 - 20x^2 + 8x - 76$
{\color{red} $$\frac{dy}{dx} = 21x^6 - 120x^5 + 80x^4 - 16x^3 + 51x^2 - 40x + 8$$}
\end{enumerate}

\section*{Problem 2: Examining Graphs of Derivatives}
 
For each expression, graph the original expression and the first two derivatives. Use these to identify:
\begin{enumerate}
	\item The location ($x$ value) of any roots of the original function or the first two derivatives.
	\item The location of any local minima or maxima in the original function or first derivative.
    \item Intervals on which the function or its first derivative is increasing or decreasing (use interval notation).
    \item The location of any inflection points in the original function
    \item Intervals on which the original function is concave up or concave down.

\end{enumerate}
\begin{enumerate}[(a)]
\item $ \displaystyle x^4-5x^3+4x^2+x+2$
{\color{red} \begin{itemize}
    \item Roots (original function): 1.589, 3.863
    \item Roots (first derivative): -0.104, 0.782, 3.073
    \item Roots (second derivative): 0.304, 2.196
    \item Local Extrema (original function): -0.104 (minimum), 0.782 (maximum), 3.073 (minimum)
    \item Local Extrema (first derivative): 0.304 (maximum), 2.196 (minimum)
    \item Increasing (original function): $(-0.104, 0.782) \cup (3.073, \infty)$
    \item Decreasing (original function): $(-\infty, -0.104) \cup (0.782, 3.073)$
    \item Increasing (first derivative): $(-\infty, 0.304) \cup (2.196, \infty)$
    \item Decreasing (first derivative): $(0.304, 2.196)$
    \item Inflection Points (original function): 0.304, 2.196
    \item Concave Up (original function): $(-\infty, 0.304) \cup (2.196, \infty)$
    \item Concave Down (original function): $(0.304, 2.196)$
\end{itemize}}
\item $ \displaystyle \tan x$, considering only the interval $[0,2\pi]$.
{\color{red} \begin{itemize}
    \item Roots (original function): $0, \pi, 2\pi$
    \item Roots (first derivative): None
    \item Roots (second derivative): $0, \pi, 2\pi$
    \item Local Extrema (original function): None
    \item Local Extrema (first derivative): $0, \pi, 2\pi$ (all minima)
    \item Increasing (original function): $[0, \frac{\pi}{2}) \cup (\frac{\pi}{2}, \frac{3\pi}{2}) \cup (\frac{3\pi}{2}, 2\pi]$
    \item Decreasing (original function): None
    \item Increasing (first derivative): $(0, \frac{\pi}{2}) \cup (\pi, \frac{3\pi}{2})$
    \item Decreasing (first derivative): $(\frac{\pi}{2}, \pi) \cup (\frac{3\pi}{2}, 2\pi)$
    \item Inflection Points (original function): $0, \pi, 2\pi$
    \item Concave Up (original function): $(0, \frac{\pi}{2}) \cup (\pi, \frac{3\pi}{2})$
    \item Concave Down (original function): $(\frac{\pi}{2}, \pi) \cup (\frac{3\pi}{2}, 2\pi)$
\end{itemize}}
\item $ \displaystyle e^x$
{\color{red} \begin{itemize}
    \item Roots (original function): None
    \item Roots (first derivative): None
    \item Roots (second derivative): None
    \item Local Extrema (original function): None
    \item Local Extrema (first derivative): None
    \item Increasing (original function): $(-\infty, \infty)$
    \item Decreasing (original function): Nowhere
    \item Increasing (first derivative):$(-\infty, \infty)$
    \item Decreasing (first derivative): Nowhere
    \item Inflection Points (original function): None
    \item Concave Up (original function): $(-\infty, \infty)$
    \item Concave Down (original function): Nowhere
\end{itemize}}
\item $ \displaystyle \sin x$, considering only the interval $[0,2\pi]$.
{\color{red} \begin{itemize}
    \item Roots (original function): $0, \pi, 2\pi$
    \item Roots (first derivative): $\frac{\pi}{2}, \frac{3\pi}{2}$
    \item Roots (second derivative): $0, \pi, 2\pi$
    \item Local Extrema (original function): $\frac{\pi}{2}$ (maximum), $\frac{3\pi}{2}$ (minimum)
    \item Local Extrema (first derivative): $0$ (maximum), $\pi$ (minimum), $2\pi$ (maximum)
    \item Increasing (original function): $(0, \frac{\pi}{2}) \cup (\frac{3\pi}{2}, 2\pi)$
    \item Decreasing (original function): $(\frac{\pi}{2}, \frac{3\pi}{2})$
    \item Increasing (first derivative):$(\pi, 2\pi)$
    \item Decreasing (first derivative): $(0, \pi)$
    \item Inflection Points (original function): $0, \pi, 2\pi$
    \item Concave Up (original function): $(\pi, 2\pi)$
    \item Concave Down (original function): $(0, \pi)$
\end{itemize}}
\end{enumerate}

Hint:  Desmos can be used to easily graph the original experession (in this case, $\sin x$ and the first two derivatives by graphing:
 	\[\sin x\]
    \[\frac{d}{dx}\sin x\]
    \[\frac{d}{dx}\frac{d}{dx}\sin x\]
    
\section*{Bonus: Fun Tasks}
\begin{enumerate}
\item Without looking it up, use Desmos to determine expressions for the first \em four \em derivatives of $\cos x$, in terms of $\sin x$ and $\cos x$.
\item A piece of wire 40 inches long is cut into two pieces.  One is bent into a circle and the other is bent into a square.  How should the pieces be cut to \em minimize \em the total area of the two shapes?
\end{enumerate}

\end{document}
