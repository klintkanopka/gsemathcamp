% ----------------------------------------------------------------
% Article Class (This is a LaTeX2e document)  ********************
% ----------------------------------------------------------------
\documentclass[11pt]{article}
\usepackage[english]{babel}
\usepackage{amsmath,amsthm, amssymb}
\usepackage{amsfonts}
\usepackage{multirow}
\usepackage{threeparttable}
\usepackage{enumerate}

%Candelaria's favorite packages
\usepackage{setspace}	% Set the spacing for the document
\usepackage{xtab}		% Use for tables
\usepackage{comment}	% Use for block comments
\usepackage{rotating}	% Helpful for rotating figures
\usepackage{lscape}	% Change a page into landscape
\usepackage{hyperref}
\hypersetup{colorlinks=false,pdfborder={0 0 0},breaklinks=true}
\usepackage{graphicx}	% Load a graphic image
%\usepackage{url}		% Use for formatting URLS (nice for NatBib also)
\usepackage{booktabs}  	% Use for nice tables/ Works well with ESTOUT
\usepackage{longtable} 	% Used to break long tables over multiple pages
\usepackage{caption}
\usepackage{subcaption} 	% Used to create SubFloats
\usepackage{listings} 	% Use to enter code blocks, like a fancy verbatim

% ADD COLOR:
\usepackage{color}
\usepackage[usenames, dvipsnames, svgnames, table]{xcolor}

% Packages for the bibliography
\usepackage[longnamesfirst]{natbib}
%\usepackage{natbib}

% Set Page Margins
%\usepackage{fullpage}
\addtolength{\oddsidemargin}{-.875in}
\addtolength{\evensidemargin}{-.875in}
\addtolength{\textwidth}{1.75in}
\addtolength{\topmargin}{-.875in}
\addtolength{\textheight}{1.75in}

% Set line and table spacing
\renewcommand{\baselinestretch}{1.0}
\renewcommand{\arraystretch}{1.1}
\setlength\abovedisplayskip{0pt}
\setlength\belowdisplayskip{0pt}

% Redefine the cite command
\renewcommand\cite{\citet}

%\setlength{\parindent}{0in}

%\setlength{\leftmargin}{0pt} \setlength{\parskip}{1.1ex} \setlength{\parindent}{1em} \setlength{\itemsep}{0pt} \setlength{\footnotesep}{10pt}
%\renewcommand{\footnoterule}{\rule{3in}{.3pt}\vspace{-.3pt}}

% THEOREMS -------------------------------------------------------
\newtheorem{thm}{Theorem}[section]
\newtheorem{cor}[thm]{Corollary}
\newtheorem{lem}[thm]{Lemma}
\newtheorem{prop}[thm]{Proposition}
\theoremstyle{definition}
\newtheorem{defn}[thm]{Definition}
\theoremstyle{remark}
\newtheorem{rem}[thm]{Remark}
%\numberwithin{equation}{section} %Number equations within section
% ----------------------------------------------------------------

% AUTHOR INFO


\title{Problem Set 1\\ Due: September 4, 2019}%

% ----------------------------------------------------------------

%%%%%%
% SHORT CITES
%%%%%%
\shortcites{ColEtAl1966}


\begin{document}

\begin{center}
{\huge Problem Set 1: Pre-Calculus}\\[5pt]
Due: September 4, 2019
%{\Large SOLUTIONS}
\end{center}

\section*{Problem 1}
Solve for x:
\begin{enumerate}
\item $4x + 5 = 17$
\item $13 - 3x = 27x - 2$
\item $0 = x^2 - x - 6$
\item $x^2 = 36$\\[5pt]
\end{enumerate}

\section*{Problem 2}
\begin{enumerate}
\item Find the equation of the line that passes through (9,3) and (4,5).
\item Do the lines $y=x$ and $y=x^2+3$ intersect?  If so, where?
\item Find the distance between $(1,2)$ and $(-4,-3)$.
\item For the function $f(x) = 3x - x^2$, evaluate $f(-2)$.
\item Plot the polynomial $y = -3 + 2x^2$. For what values of x is this function increasing? Decreasing?\\[5pt]
\end{enumerate}

\section*{Problem 3}
Evaluate:
\begin{enumerate}
\item $$\sum_{x=1}^{5}x^2-x$$
\item $$\sum_{x=0}^{10}(-1)^{x} (3x + 2) $$
\item $$\sum_{x=1}^{5} 5$$
\item $$\prod_{x=0}^{5}3x^3-x$$
\item $$\prod_{x=1}^{3}(-1)^{x} (x - 2x^2) $$
\item $$\prod_{x=1}^{5} 5$$
\end{enumerate}

\section*{Problem 4}
Explain in your own words what is meant by the equation
\[\lim_{x\to 2} f(x) = 5\]
Is it possible for this statement to be true and yet have $f(2) = 3$? Explain.\\[5pt]

\section*{Problem 5}
The \textbf{greatest} integer function is defined by $\lfloor x \rfloor=$ the largest integer that is less than or equal to $x$. (For instance, $\lfloor 4 \rfloor=4$, $\lfloor 4.8 \rfloor=4$, $\lfloor \pi \rfloor=3$, $\lfloor \sqrt{2} \rfloor=1$, $\lfloor -\frac{1}{2} \rfloor= -1.)$ Show that $\lim_{x \to 3} \lfloor x \rfloor $ does not exist. Hint: You may want to draw a graph. \\[5pt]
Sidenote: The greatest integer function is also called the \em floor \em function - it just rounds down to the nearest integer. Contrast this with the \em ceiling \em function, which just rounds up to the nearest integer. 


\section*{Bonus}
Solve for $n$:
$$\sum_{x=1}^{n} (5x^2 -7x + 1) = 7871 $$


\end{document}



